\documentclass[margin=2mm]{standalone}
\usepackage{amsmath, amssymb, amsfonts}
\usepackage{tudscrcolor} 
\usepackage{pgfplots}
\usepgfplotslibrary{fillbetween} 
\pgfplotsset{
  compat=1.10,% mit writeLaTeX bisher noch nicht möglich
  flaeche/.style={draw=none,fill=black,fill opacity=0.2},
  every axis/.append style={
  	axis x line=middle,    % put the x axis in the middle
  	axis y line=middle,    % put the y axis in the middle
  	axis line style={->},  % arrows on the axis
  }
}

\renewcommand{\Re}{\mathrm{Re}}
\renewcommand{\Im}{\mathrm{Im}}

\usepackage{opensans}
%\usepackage{sfmath}

\begin{document} 

\begin{tikzpicture} 
	\begin{axis}[%
		xmin = -0.1,
		xmax =  7,
		ymin = -1.2,
		ymax =  1.2,
		axis equal,
%		ticks=none,
		grid=none,
		xlabel=$$,
		xlabel style={below, anchor=south east,inner xsep=0pt},
		ylabel=$$,
		ylabel style={above,anchor=north west,inner ysep=0pt},
		xtick = {3.14,6.28},
		xticklabels = {$\pi$, $2\pi$},
%		tick style={ultra thick, black}, 
		ytick = {0.5, -0.866},
		yticklabels= {$\frac{1}{2}$, $-\frac{1}{2}\sqrt{3}$}, 
	]
	
		\addplot [domain=0:6.28, smooth, color=cdblue, thick, samples=100, variable=\x] {sin(deg(x))};
		\addplot [domain=0:6.28, smooth, color=cdgreen, thick, samples=100, variable=\x] {cos(deg(x))};
		
		\addplot [domain=0:6.28, smooth, color=cdgreen, dashed, variable=\x] {0.5};
		\addplot [domain=0:6.28, smooth, color=cdblue, dashed, variable=\x] {-0.5*sqrt(3)};
		
		\legend{$\sin(x)$, $\cos(x)$}
		
%		\addplot [only marks, mark=o, mark size = 2pt, draw=black]
%		coordinates { (5/3 * pi,0.5) (1/3 * pi, 0.5)};
%		\addplot [only marks, mark=o, mark size = 2pt, draw=black]
%		coordinates { (5/3 * pi, {-1/2 * sqrt(3) }) (4/3 * pi , {-1/2 * sqrt(3)} ) };
		
%		\node[align=left] at (axis cs: 1.5, -2) {$r = |z| = 2$};
%		\node[align=left] at (axis cs: 5, -2) {$\sin(\varphi) = \frac{\Im(z)}{r} = - \frac{\sqrt{3}}{2}$ \\ $\cos(\varphi) = \frac{\Re(z)}{r} = \phantom{-} \frac{1}{2}$ };
	\end{axis} 

	\node[above,font=\bfseries] at (current bounding box.north) {$z = 1 - \sqrt{3} i = r \cdot e^{i \varphi}$};
	
\end{tikzpicture}

\end{document}