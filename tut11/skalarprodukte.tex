\documentclass{beamer}
\usepackage{../tut-slides-LA}
\usepackage{../mathoperatorsLA}

\usepackage{lmodern}
\usepackage{amsmath,amssymb}
\usepackage{wasysym}
\usepackage{stmaryrd}
\usepackage{enumerate}
%\usepackage[inline]{enumitem} 		%customize label
%\newcommand{\labelitemi}{\raisebox{1pt}{\scalebox{.9}{$\blacktriangleright$}}}
%\newcommand{\labelitemii}{$\vartriangleright$}
%\newcommand{\labelitemiii}{--}
\setbeamertemplate{itemize item}{\raisebox{1pt}{\scalebox{.9}{$\blacktriangleright$}}}
\setbeamertemplate{itemize subitem}{$\vartriangleright$}

\usepackage{booktabs}
\usepackage{tabularx}
\usepackage{tabu}
\newcommand*\head{\rowfont{\bfseries}}
\newcommand*{\tw}{\rowfont{\ttfamily}}
\renewcommand{\tabularxcolumn}[1]{>{\hspace{0pt}}m{#1}}
\usepackage{multirow}

\usepackage{cancel}

\usepackage{empheq}
\newcommand*\widefbox[1]{\fbox{\hspace{2em} #1 \hspace{2em}}}

\usepackage{tcolorbox}
\newtcolorbox{mymathbox}[1][]{colback=white, sharp corners, #1}

\usepackage{xcolor}
\usepackage{listings}
\lstset{numbers=left,
	numberstyle=\tiny,
	breaklines=true,
	backgroundcolor=\color{cdgray!20},
	numbersep=5pt,
	language=C,
	tabsize=2,
	basicstyle=\footnotesize\ttfamily,
	showstringspaces=false}

\DeclareMathOperator{\ack}{\mathbf{ack}}
\usepackage{MnSymbol}

\usepackage{csquotes}
\usepackage{ragged2e}


\begin{document}
	\title{Lineare Algebra}
	\subtitle{Übung 11: Skalarprodukte}
	\author{Eric Kunze}
	\email{eric.kunze@mailbox.tu-dresden.de}
	\city{TU Dresden}
%	\institute{Lehrstuhl für Grundlagen der Programmierung}
	\titlegraphic{\includegraphics[width=2cm]{../TUD-white.pdf}}
	\date{19.01.2021}

	\maketitle

%%%%%%%%%%%%%%%%%%%%%%%%%%%%%%%%%%%%%%%%%%%%%%%%%%%%%%%%%%%%%%%%%%%%%%%%%%%%%

\begin{frame} \frametitle{Skalarprodukt}
    Sei $V$ ein Vektorraum über einem Körper $K$.

    Eine Abbildung $\bullet : V \times V \to K$ heißt \emph{Skalarprodukt}, wenn folgende Eigenschaften erfüllt sind:
    \begin{itemize}
        \item $\bullet$ ist \emph{bilinear},
        \item $\bullet$ ist \emph{symmetrisch},
        \item $\bullet$ ist \emph{positiv definit}.
    \end{itemize}

    Manchmal schreibt man für Vektoren $u,v \in V$ statt $u \bullet v$ auch $\langle u , v \rangle$ oder $(u,v)$.
    
    Einen Vektorraum $V$ gemeinsam mit einem Skalarprodukt $\bullet$ nennt man \emph{euklidischen Vektorraum} und schreibt $(V, \bullet)$ für diese Paarung.
\end{frame}

\begin{frame} \frametitle{Bilinearität von $\bullet$}
    Die Abbildung $\bullet$ ist \enquote{linear in jedem Argument}:
    \begin{enumerate}
        \item Linearität im ersten Argument:
        \begin{itemize}
            \item additiv: $(u_1 + u_2) \bullet v = (u_1 \bullet v) + (u_2 \bullet v)$
            \item homogen: $(k u) \bullet v = k (u \bullet v)$
        \end{itemize}
        \smallskip oder in einem Schritt \smallskip
        \begin{itemize}
            \item $(k_1 u_1 + k_2 u_2) \bullet v = k_1 (u_1 \bullet v) + k_2 (u_2 \bullet v)$
        \end{itemize}
        \medskip
        \item Linearität im zweiten Argument:
        \begin{itemize}
            \item additiv: $u \bullet (v_1 + v_2) = (u \bullet v_1) + (u \bullet v_2)$
            \item homogen: $u \bullet (\ell v) = \ell \ (u \bullet v)$
        \end{itemize}
        \smallskip oder in einem Schritt \smallskip
        \begin{itemize}
            \item $u \bullet (\ell_1 v_1 + \ell_2 v_2) = \ell_1 (u \bullet v_1) + \ell_2 (u \bullet v_2)$
        \end{itemize}
    \end{enumerate}
\end{frame}

\begin{frame} \frametitle{Bilinearität von $\bullet$}
    Alternativ kann man auch alles in einem Rutsch erledigen und zeigen, dass
    \begin{align*}
        &(k_1 u_1 + k_2 u_2) \bullet (\ell_1 v_1 + \ell_2 v_2) \\
        = \enskip
        &
        k_1 \Big(u_1 \bullet (\ell_1 v_1 + \ell_2 v_2) \Big) + k_2 \Big(u_2 \bullet (\ell_1 v_1 + \ell_2 v_2) \Big) \\
        = \enskip
        &
        k_1 \ell_1 (u_1 \bullet v_1) + k_1 \ell_2 (u_1 \bullet v_2) + k_2 \ell_1 (u_2 \bullet v_1) + k_2 \ell_2 (u_2 \bullet v_2)
    \end{align*}
    gilt.
\end{frame}

\begin{frame} \frametitle{Bilinearität $\leftrightarrow$ Linearität}
    \justifying \small
    Wir können Bilinearität auch mit der bekanten Linearität in Verbindung bringen. Definieren wir dazu die Abbildung
    \begin{equation*}
        s_v : V \to K \quad \text{mit} \quad s_v(u) \defeq u \bullet v \enskip \text{.}
    \end{equation*}
    Dann ist Linearität im ersten Argument äquivalent zur Linearität von $s_v$, denn
    \begin{equation*}
        s_v(k_1 u_1 + k_2 u_2) = (k_1 u_1 + k_2 u_2) \bullet v \enskip \text{.}
    \end{equation*}
    \pause
    Definieren wir analog dazu
    \begin{equation*}
        s_u : V \to K \quad \text{mit} \quad s_u(v) \defeq u \bullet v \enskip \text{,}
    \end{equation*}
    dann ist wieder Linearität im zweiten Argument äquivalent zur Linearität von $s_u$, denn
    \begin{equation*}
        s_u(\ell_1 v_1 + \ell_2 v_2) = u \bullet (\ell_1 v_1 + \ell_2 v_2) \enskip \text{.}
    \end{equation*}
    Linearität von Abbildungen zwischen Vektorräumen haben wir in Übung 5 bereits studiert. Man beachte hier, dass jeder Körper auch einen Vektorraum bildet.
\end{frame}

\begin{frame} \frametitle{Beispiel: Standardskalarprodukt}
	\justifying \small
	Aus der Schule bekannt ist das \emph{Standardskalarprodukt} von $u,v \in \Rn$ durch
	\begin{equation}
		u \bullet v = u_1 v_1 + u_2 v_2 + \dots u_n v_n \enskip \text{.}
		\tag{$\star$} \label{eq: standardskalarprodukt}
	\end{equation}
	Mit Matrixmultiplikation können wir auch \fbox{$u \bullet v = u^\top v$} schreiben.
	\begin{itemize}
		\item \emph{Bilinearität} folgt aus den Eigenschaften der Matrixmultiplikation.
		\item \emph{Symmetrie:} klar mit \eqref{eq: standardskalarprodukt}.
		\item \emph{Positive Definitheit:} $u \bullet u = \underbrace{u_1^2}_{\ge 0} + \underbrace{u_2^2}_{\ge 0} + \dots + \underbrace{u_n^2}_{\ge 0} \ge 0$ und $u \bullet u = 0$ kann nur auftreten, wenn jeder Summand schon Null ist, d.h. $u_1 = 0 \land u_2 = 0 \land \dots \land u_n = 0$, was gerade $u=0$ entspricht.
	\end{itemize}
\end{frame}

\begin{frame} \frametitle{Beispiel: Skalarprodukt mit Matrix}
	\justifying \small
	Anstelle von $u^\top v$ können wir etwas allgemeiner $u \bullet v$ definieren als 
	\begin{equation*}
		u \bullet v = u^\top M v
	\end{equation*}
	mit einer Matrix $M$. Welche Bedingungen muss $M$ erfüllen, damit $\bullet$ ein Skalarprodukt wird?
	\begin{itemize}
		\item \emph{Bilinearität} folgt weiter aus den Eigenschaften der Matrixmultiplikation (vgl. auch Ü106).
		\item \emph{Symmetrie:} Dafür muss $M$ symmetrisch sein, d.h. $M^\top = M$. 
		\begin{equation*}
			u \bullet v = \underbrace{u^\top M v}_{\in \R} 
			= \brackets{u^\top M v}^\top 
			= v^\top M^\top \brackets{u^\top}^\top 
			= v^\top \underbrace{M^\top}_{\mathrlap{\hspace{-1em}= M\text{, wenn $M$ symm.}}} u
			= v \bullet u
		\end{equation*}
		\item \emph{Positive Definitheit:} Hier muss man die positive Definitheit direkt zeigen, d.h. $u^\top M u \ge 0$ und $u^\top M u = 0 \equivalent u = 0$ oder gleichwertig dazu $u^\top M u > 0$ für alle $u \neq 0$.
	\end{itemize}
\end{frame}

\begin{frame} \frametitle{Beispiel: Integrale?!}
	Wir haben aber auch abstraktere Vektorräume kennengelernt, z.B. den Vektorraum der Abbildungen $V = \menge{f : \R \to \R}$. Wie können wir dort ein Skalarprodukt definieren?
	
	Man kann zeigen, dass für $f,g \in V$ der Ausdruck
	\begin{equation*}
		f \bullet g \defeq \int_0^1 f(x) \ g(x) \enskip dx
	\end{equation*}
	ein Skalarprodukt $\bullet$ auf $V$ definiert.
\end{frame}
% Beispiele: Standardskalarprodukt, von Matrix induziertes SP, SP in Abb(R,R)
% Längenmessung, Norm
% Winkelmessung, Orthogonalität

\begin{frame} \frametitle{Längenmessung}
	Ein Skalarprodukt $\bullet$ induziert durch
	\begin{equation*}
		\norm{v} \defeq \sqrt{v \bullet v}
	\end{equation*}
	eine \emph{Norm} auf $V$. 
	
	Damit können wir \emph{Längen} von Vektoren messen.
\end{frame}

\begin{frame} \frametitle{Winkelmessung}
	Sei $(V,\bullet)$ ein euklidischer Vektorraum. Der \emph{Winkel} $\phi \in [0,2\pi)$ zwischen zwei Vektoren $u,v \in V$ ist bestimmt durch
	\begin{equation*}
		\cos(\phi) = \frac{u \bullet v}{\norm{u} \norm{v}} \enskip \text{.}
	\end{equation*}

	Zwei Vektoren $u,v \in V$ heißen \emph{orthogonal} (in Zeichen $u \perp v$), wenn 
	\begin{equation*}
		u \bullet v = 0
	\end{equation*}
	gilt. Dann ist nämlich $\cos(\phi) = \frac{u \bullet v}{\norm{u} \norm{v}} = 0$ und damit $\phi \in \menge{\tfrac{\pi}{2}, \tfrac{3\pi}{2}}$.
\end{frame}

\end{document}